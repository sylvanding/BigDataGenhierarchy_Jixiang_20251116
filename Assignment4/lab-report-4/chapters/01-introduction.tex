% Chapter 1: Introduction
\section{引言}

\subsection{研究背景与意义}

度量空间(Metric Space)是一种通用的数据抽象方式,可以涵盖向量、字符串、图像、生物序列等多种数据类型。在度量空间中,数据对象之间的相似性通过满足正定性、对称性和三角不等式的距离函数来度量。相似性查询是度量空间数据管理中最基本的操作之一,包括范围查询(Range Query)和k近邻查询(kNN Query)。

在Assignment 3中,我们实现了两种基础的树状索引——GH树和VP树,它们分别使用2个和1个支撑点(Pivot)进行数据划分。然而,这些基础索引在面对大规模或高维数据时,剪枝效果可能不足。\textbf{多Pivot索引}的核心思想是:

\begin{quote}
    使用更多的pivot可以获得更多的距离信息,从而实现更精细的数据划分和更有效的查询剪枝。
\end{quote}

本次实验深入研究三种使用3个pivot的多Pivot树状索引结构,探索不同的划分策略对索引性能的影响。

\subsection{任务回顾与目标}

\subsubsection{前序工作简述}

在Assignment 1和2中,我们建立了度量空间数据管理的基础设施:
\begin{itemize}
    \item 核心抽象类:\texttt{MetricSpaceData}和\texttt{MetricFunction}
    \item 具体数据类型:向量数据(\texttt{VectorData})和蛋白质序列(\texttt{ProteinData})
    \item Pivot Table索引:利用支撑点预计算距离实现查询剪枝
\end{itemize}

在Assignment 3中,我们实现了两种树状索引:
\begin{itemize}
    \item \textbf{GH树}:使用2个pivot进行超平面划分
    \item \textbf{VP树}:使用1个pivot进行球形划分
\end{itemize}

\subsubsection{本次任务目标}

本次Assignment 4的核心目标是实现和对比分析\textbf{三种使用3个pivot的度量空间树状索引结构}:

\begin{enumerate}
    \item \textbf{3-pivot MVPT(多优势点树)}:VP树的扩展,使用3个支撑点进行嵌套球形划分,产生$2^3=8$个子区域
    \item \textbf{3-pivot CGHT(完全广义超平面树)}:GH树的扩展,根据距离差进行多路划分,充分利用pivot对之间的距离差信息
    \item \textbf{3-pivot 完全线性划分树}:在支撑点空间中使用线性边界进行数据划分
\end{enumerate}

具体任务包括:
\begin{enumerate}
    \item \textbf{代码实现}:实现三种多Pivot树索引的数据结构、批建算法和范围查询
    \item \textbf{正确性验证}:通过与线性扫描对比验证实现的正确性
    \item \textbf{理论分析}:从理论角度对比分析三种索引的区别、联系和优缺点
    \item \textbf{实验分析}:在多种数据集上进行性能对比实验
\end{enumerate}

\subsection{实验环境}

本实验的软硬件环境如表\ref{tab:environment}所示。

\begin{table}[htbp]
    \centering
    \caption{实验环境配置}
    \label{tab:environment}
    \begin{tabular}{ll}
        \toprule
        \textbf{项目} & \textbf{配置} \\
        \midrule
        操作系统 & Windows 11 \\
        CPU & Intel Core \\
        内存 & 16GB \\
        编程语言 & Java 11 \\
        构建工具 & Maven 3.8+ \\
        IDE & VS Code / Cursor \\
        \bottomrule
    \end{tabular}
\end{table}

\subsection{报告结构}

本报告的组织结构如下:

\begin{itemize}
    \item \textbf{第2章 理论基础}:介绍度量空间与支撑点空间、MVP树、完全广义超平面树和完全线性划分的原理
    \item \textbf{第3章 算法实现}:详细描述系统架构和三种索引的核心算法实现
    \item \textbf{第4章 正确性验证}:通过测试数据集和与线性扫描对比验证实现的正确性
    \item \textbf{第5章 理论对比分析}:从理论角度对比分析三种索引的划分方式、剪枝能力和复杂度
    \item \textbf{第6章 实验对比分析}:在多种数据集上进行性能对比实验并分析结果
    \item \textbf{第7章 总结与展望}:总结本次工作,展望未来研究方向
\end{itemize}

本项目的完整代码已上传至GitHub仓库:

\url{https://github.com/sylvanding/BigDataGenhierarchy_Jixiang_20251116}
