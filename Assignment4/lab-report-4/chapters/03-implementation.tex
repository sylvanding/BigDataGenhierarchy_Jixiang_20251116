% Chapter 3: Implementation
\section{算法实现}

\subsection{系统架构}

\subsubsection{Assignment 4新增模块}

在Assignment 3的基础上,我们扩展系统架构以支持多Pivot树索引。图\ref{fig:architecture-a4}展示了扩展后的系统模块关系。

\begin{figure}[htbp]
    \centering
    \begin{tikzpicture}[
        node distance=1cm,
        box/.style={rectangle, draw, rounded corners, minimum width=2.2cm, minimum height=0.7cm, text centered, font=\footnotesize},
        arrow/.style={->, thick}
    ]
        % Tree Index base
        \node[box, fill=orange!20] (treeindex) {TreeIndex};
        
        % Assignment 3 trees
        \node[box, fill=yellow!20, below left=1cm and 0.5cm of treeindex] (ghtree) {GHTree};
        \node[box, fill=yellow!20, right=0.2cm of ghtree] (vptree) {VPTree};
        
        % Assignment 4 trees
        \node[box, fill=red!20, below right=1cm and -0.5cm of treeindex] (mvptree) {MVPTree};
        \node[box, fill=red!20, right=0.2cm of mvptree] (cghtree) {CGHTree};
        \node[box, fill=red!20, right=0.2cm of cghtree] (lptree) {LPTree};
        
        % Common components
        \node[box, fill=green!20, above=1.5cm of treeindex] (config) {TreeConfig};
        \node[box, fill=green!20, right=0.3cm of config] (selector) {MultiPivotSelector};
        \node[box, fill=green!20, right=0.3cm of selector] (controller) {HeightController};
        
        % Arrows
        \draw[arrow] (ghtree) -- (treeindex);
        \draw[arrow] (vptree) -- (treeindex);
        \draw[arrow] (mvptree) -- (treeindex);
        \draw[arrow] (cghtree) -- (treeindex);
        \draw[arrow] (lptree) -- (treeindex);
        
        \draw[arrow,dashed] (config) -- (treeindex);
        \draw[arrow,dashed] (selector) -- (treeindex);
        \draw[arrow,dashed] (controller) -- (treeindex);
        
        % Labels
        \node[below=0.1cm of ghtree, font=\scriptsize] {Assignment 3};
        \node[below=0.1cm of cghtree, font=\scriptsize] {Assignment 4 (NEW)};
    \end{tikzpicture}
    \caption{Assignment 4系统架构扩展}
    \label{fig:architecture-a4}
\end{figure}

表\ref{tab:new-modules}展示了Assignment 4新增的主要模块。

\begin{table}[htbp]
    \centering
    \caption{Assignment 4新增模块}
    \label{tab:new-modules}
    \begin{tabular}{ll}
        \toprule
        \textbf{Module/Class} & \textbf{Description} \\
        \midrule
        \texttt{MultiPivotSelector} & Multi-pivot selection with FFT/RANDOM/MAX\_SPREAD \\
        \texttt{mvptree.MVPTree} & 3-pivot MVP tree implementation \\
        \texttt{mvptree.MVPInternalNode} & MVP tree internal node \\
        \texttt{cght.CGHTree} & 3-pivot CGH tree implementation \\
        \texttt{cght.CGHInternalNode} & CGH tree internal node \\
        \texttt{linearpartition.LinearPartitionTree} & 3-pivot linear partition tree \\
        \texttt{linearpartition.LPInternalNode} & Linear partition tree internal node \\
        \bottomrule
    \end{tabular}
\end{table}

\subsection{3-pivot MVPT实现}

\subsubsection{数据结构}

MVP树内部节点的核心数据结构如下:

\begin{lstlisting}[caption=MVPInternalNode数据结构]
public class MVPInternalNode extends InternalNode {
    private MetricSpaceData[] pivots;      // 3 pivots
    private TreeNode[] children;            // 8 children
    private double[] splitRadius;           // 3 split radii (medians)
    private double[][] lowerBound;          // [8][3] lower bounds
    private double[][] upperBound;          // [8][3] upper bounds
    
    public int getChildIndex(double d1, double d2, double d3) {
        int idx = 0;
        if (d1 > splitRadius[0]) idx |= 1;
        if (d2 > splitRadius[1]) idx |= 2;
        if (d3 > splitRadius[2]) idx |= 4;
        return idx;
    }
}
\end{lstlisting}

\subsubsection{批建算法}

MVP树批建算法的核心逻辑如算法\ref{alg:mvp-build}所示。

\begin{algorithm}[htbp]
    \caption{MVP树批建算法}
    \label{alg:mvp-build}
    \begin{algorithmic}[1]
        \REQUIRE 数据集$S$,当前深度$depth$
        \ENSURE MVP树节点
        \IF{$|S| \leq maxLeafSize$ \AND $depth \geq minTreeHeight$}
            \RETURN LeafNode($S$, $depth$)
        \ENDIF
        \STATE $(p_1, p_2, p_3) \leftarrow$ SelectThreePivots($S$)
        \STATE 计算所有数据到各pivot的距离
        \STATE $\mu_1, \mu_2, \mu_3 \leftarrow$ 各pivot维度的中位数
        \STATE 初始化8个子集 $partitions[0..7]$
        \FOR{each $x$ in $S - \{p_1, p_2, p_3\}$}
            \STATE $idx \leftarrow$ GetChildIndex($d(x,p_1), d(x,p_2), d(x,p_3)$)
            \STATE $partitions[idx]$.add($x$)
        \ENDFOR
        \STATE 计算每个子集的距离范围 $[L_{i,j}, U_{i,j}]$
        \FOR{$i = 0$ to $7$}
            \STATE $children[i] \leftarrow$ BuildMVPTree($partitions[i]$, $depth+1$)
        \ENDFOR
        \RETURN MVPInternalNode($pivots$, $children$, $bounds$)
    \end{algorithmic}
\end{algorithm}

\subsubsection{范围查询算法}

MVP树范围查询利用距离范围进行剪枝:

\begin{lstlisting}[caption=MVP树范围查询核心实现]
private void rangeQueryRecursive(TreeNode node, 
        MetricSpaceData q, double r, List<MetricSpaceData> result) {
    if (node.isLeaf()) {
        // Linear scan leaf data
        for (MetricSpaceData obj : ((LeafNode)node).getData()) {
            if (metric.getDistance(q, obj) <= r) {
                result.add(obj);
            }
        }
        return;
    }
    
    MVPInternalNode internal = (MVPInternalNode) node;
    double[] dq = new double[3];
    for (int j = 0; j < 3; j++) {
        dq[j] = metric.getDistance(q, internal.getPivot(j));
        if (dq[j] <= r) result.add(internal.getPivot(j));
    }
    
    // Check each child
    for (int i = 0; i < 8; i++) {
        if (internal.getChild(i) == null) continue;
        
        boolean canPrune = false;
        boolean fullyContained = false;
        
        for (int j = 0; j < 3; j++) {
            double L = internal.getLowerBound(i, j);
            double U = internal.getUpperBound(i, j);
            
            // Exclusion rule
            if (dq[j] + r < L || dq[j] - r > U) {
                canPrune = true;
                break;
            }
            // Containment rule
            if (dq[j] + U <= r) {
                fullyContained = true;
                break;
            }
        }
        
        if (fullyContained) {
            collectAllData(internal.getChild(i), result);
        } else if (!canPrune) {
            rangeQueryRecursive(internal.getChild(i), q, r, result);
        }
    }
}
\end{lstlisting}

\subsection{3-pivot CGHT实现}

\subsubsection{数据结构}

CGH树内部节点基于距离差进行划分:

\begin{lstlisting}[caption=CGHInternalNode数据结构]
public class CGHInternalNode extends InternalNode {
    private MetricSpaceData[] pivots;       // 3 pivots
    private TreeNode[] children;             // 4 children
    private double[][] delta12Range;         // [4][2]: min, max of delta12
    private double[][] delta13Range;         // [4][2]: min, max of delta13
    
    public int getChildIndex(double d1, double d2, double d3) {
        double delta12 = d1 - d2;
        double delta13 = d1 - d3;
        int idx = 0;
        if (delta12 >= 0) idx |= 1;
        if (delta13 >= 0) idx |= 2;
        return idx;
    }
}
\end{lstlisting}

\subsubsection{批建算法}

CGH树批建算法如算法\ref{alg:cgh-build}所示。

\begin{algorithm}[htbp]
    \caption{CGH树批建算法}
    \label{alg:cgh-build}
    \begin{algorithmic}[1]
        \REQUIRE 数据集$S$,当前深度$depth$
        \ENSURE CGH树节点
        \IF{$|S| \leq maxLeafSize$ \AND $depth \geq minTreeHeight$}
            \RETURN LeafNode($S$, $depth$)
        \ENDIF
        \STATE $(p_1, p_2, p_3) \leftarrow$ SelectThreePivots($S$)
        \STATE 初始化4个子集 $partitions[0..3]$
        \FOR{each $x$ in $S - \{p_1, p_2, p_3\}$}
            \STATE $d_1, d_2, d_3 \leftarrow d(x, p_1), d(x, p_2), d(x, p_3)$
            \STATE $\delta_{12} \leftarrow d_1 - d_2$, $\delta_{13} \leftarrow d_1 - d_3$
            \STATE $idx \leftarrow$ ($\delta_{12} \geq 0$ ? 1 : 0) + ($\delta_{13} \geq 0$ ? 2 : 0)
            \STATE $partitions[idx]$.add($x$)
        \ENDFOR
        \STATE 计算每个子集的$\delta_{12}$和$\delta_{13}$范围
        \FOR{$i = 0$ to $3$}
            \STATE $children[i] \leftarrow$ BuildCGHTree($partitions[i]$, $depth+1$)
        \ENDFOR
        \RETURN CGHInternalNode($pivots$, $children$, $deltaRanges$)
    \end{algorithmic}
\end{algorithm}

\subsubsection{范围查询算法}

CGH树范围查询基于GH树剪枝规则的扩展:

\begin{lstlisting}[caption=CGH树范围查询核心实现]
private void rangeQueryRecursive(TreeNode node,
        MetricSpaceData q, double r, List<MetricSpaceData> result) {
    if (node.isLeaf()) {
        for (MetricSpaceData obj : ((LeafNode)node).getData()) {
            if (metric.getDistance(q, obj) <= r) {
                result.add(obj);
            }
        }
        return;
    }
    
    CGHInternalNode internal = (CGHInternalNode) node;
    double d1 = metric.getDistance(q, internal.getPivot(0));
    double d2 = metric.getDistance(q, internal.getPivot(1));
    double d3 = metric.getDistance(q, internal.getPivot(2));
    
    // Check pivots
    if (d1 <= r) result.add(internal.getPivot(0));
    if (d2 <= r) result.add(internal.getPivot(1));
    if (d3 <= r) result.add(internal.getPivot(2));
    
    double deltaQ12 = d1 - d2;
    double deltaQ13 = d1 - d3;
    
    // Check each child
    for (int i = 0; i < 4; i++) {
        if (internal.getChild(i) == null) continue;
        
        double[] range12 = internal.getDelta12Range(i);
        double[] range13 = internal.getDelta13Range(i);
        
        // Extended GH-tree pruning rule
        boolean canPrune = false;
        
        // Check delta12 range
        if (deltaQ12 - 2*r > range12[1] || deltaQ12 + 2*r < range12[0]) {
            canPrune = true;
        }
        // Check delta13 range
        if (deltaQ13 - 2*r > range13[1] || deltaQ13 + 2*r < range13[0]) {
            canPrune = true;
        }
        
        if (!canPrune) {
            rangeQueryRecursive(internal.getChild(i), q, r, result);
        }
    }
}
\end{lstlisting}

\subsection{完全线性划分树实现}

\subsubsection{数据结构}

线性划分树在支撑点空间中使用正交划分:

\begin{lstlisting}[caption=LinearPartitionInternalNode数据结构]
public class LinearPartitionInternalNode extends InternalNode {
    private MetricSpaceData[] pivots;       // 3 pivots
    private TreeNode[] children;             // 8 children
    private double[] splitThreshold;         // 3 thresholds (medians)
    private double[][] lowerBound;           // [8][3] coordinate lower bounds
    private double[][] upperBound;           // [8][3] coordinate upper bounds
    
    public int getChildIndex(double d1, double d2, double d3) {
        int idx = 0;
        if (d1 > splitThreshold[0]) idx |= 1;
        if (d2 > splitThreshold[1]) idx |= 2;
        if (d3 > splitThreshold[2]) idx |= 4;
        return idx;
    }
}
\end{lstlisting}

\subsubsection{批建算法}

线性划分树批建算法与MVP树类似,但在支撑点空间中进行划分。

\subsubsection{范围查询算法}

线性划分树范围查询在支撑点空间中判断立方体相交:

\begin{lstlisting}[caption=线性划分树范围查询核心实现]
private void rangeQueryRecursive(TreeNode node,
        MetricSpaceData q, double r, List<MetricSpaceData> result) {
    if (node.isLeaf()) {
        for (MetricSpaceData obj : ((LeafNode)node).getData()) {
            if (metric.getDistance(q, obj) <= r) {
                result.add(obj);
            }
        }
        return;
    }
    
    LPInternalNode internal = (LPInternalNode) node;
    double[] dq = new double[3];
    for (int j = 0; j < 3; j++) {
        dq[j] = metric.getDistance(q, internal.getPivot(j));
        if (dq[j] <= r) result.add(internal.getPivot(j));
    }
    
    // Query region is a cube: [dq[j]-r, dq[j]+r] for each j
    for (int i = 0; i < 8; i++) {
        if (internal.getChild(i) == null) continue;
        
        boolean canPrune = false;
        for (int j = 0; j < 3; j++) {
            double L = internal.getLowerBound(i, j);
            double U = internal.getUpperBound(i, j);
            
            // Check if query cube intersects with child region
            if (dq[j] + r < L || dq[j] - r > U) {
                canPrune = true;
                break;
            }
        }
        
        if (!canPrune) {
            rangeQueryRecursive(internal.getChild(i), q, r, result);
        }
    }
}
\end{lstlisting}

\subsection{统一的支撑点选择策略}

为了公平对比,三种索引使用统一的\texttt{MultiPivotSelector}选择支撑点:

\begin{lstlisting}[caption=MultiPivotSelector实现]
public class MultiPivotSelector {
    public enum SelectionStrategy { RANDOM, FFT, MAX_SPREAD }
    
    public MetricSpaceData[] selectPivots(
            List<? extends MetricSpaceData> data, 
            int numPivots,
            MetricFunction metric) {
        switch (strategy) {
            case RANDOM:
                return selectRandomPivots(data, numPivots);
            case FFT:
                return selectFFTPivots(data, numPivots, metric);
            case MAX_SPREAD:
                return selectMaxSpreadPivots(data, numPivots, metric);
            default:
                return selectFFTPivots(data, numPivots, metric);
        }
    }
    
    private MetricSpaceData[] selectFFTPivots(
            List<? extends MetricSpaceData> data, 
            int numPivots,
            MetricFunction metric) {
        MetricSpaceData[] pivots = new MetricSpaceData[numPivots];
        
        // First pivot: random
        pivots[0] = data.get(random.nextInt(data.size()));
        
        // Subsequent pivots: farthest from selected
        for (int i = 1; i < numPivots; i++) {
            double maxMinDist = -1;
            MetricSpaceData farthest = null;
            
            for (MetricSpaceData obj : data) {
                double minDist = Double.MAX_VALUE;
                for (int j = 0; j < i; j++) {
                    double d = metric.getDistance(obj, pivots[j]);
                    minDist = Math.min(minDist, d);
                }
                if (minDist > maxMinDist) {
                    maxMinDist = minDist;
                    farthest = obj;
                }
            }
            pivots[i] = farthest;
        }
        return pivots;
    }
}
\end{lstlisting}

\subsection{树配置与使用示例}

三种索引使用统一的配置方式:

\begin{lstlisting}[caption=多Pivot树索引使用示例]
// Configuration
TreeConfig config = new TreeConfig.Builder()
    .maxLeafSize(50)
    .minTreeHeight(2)
    .pivotStrategy(TreeConfig.PivotSelectionStrategy.FFT)
    .randomSeed(42)
    .verbose(false)
    .build();

// Create multi-pivot selector
MultiPivotSelector selector = new MultiPivotSelector(
    MultiPivotSelector.SelectionStrategy.FFT, 42);

// Build MVP Tree
MVPTree mvpTree = new MVPTree(config, selector);
mvpTree.buildIndex(dataset, new MinkowskiDistance(2));

// Build CGH Tree
CGHTree cghTree = new CGHTree(config, selector);
cghTree.buildIndex(dataset, new MinkowskiDistance(2));

// Build Linear Partition Tree
LinearPartitionTree lpTree = new LinearPartitionTree(config, selector);
lpTree.buildIndex(dataset, new MinkowskiDistance(2));

// Range Query
List<MetricSpaceData> results = mvpTree.rangeQuery(queryObj, radius);

// kNN Query
List<MetricSpaceData> knnResults = cghTree.knnQuery(queryObj, k);
\end{lstlisting}
