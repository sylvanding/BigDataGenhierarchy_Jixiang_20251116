% Chapter 7: Conclusion
\section{总结与展望}

\subsection{工作总结}

本次Assignment 4成功完成了以下工作:

\subsubsection{代码实现}

我们实现了三种使用3个pivot的多Pivot树状索引:

\begin{enumerate}
    \item \textbf{3-pivot MVP树}:基于嵌套球形划分,产生8个子区域,具有包含规则
    \item \textbf{3-pivot CGH树}:基于距离差的超平面划分,产生4个子区域
    \item \textbf{3-pivot 完全线性划分树}:在支撑点空间中正交划分,产生8个子区域
\end{enumerate}

三种索引都实现了:
\begin{itemize}
    \item 统一的\texttt{Index}接口
    \item 批建算法(\texttt{buildIndex})
    \item 范围查询(\texttt{rangeQuery})
    \item kNN查询(\texttt{knnQuery})
    \item 统计信息收集
\end{itemize}

\subsubsection{正确性验证}

通过以下方式验证了实现的正确性:
\begin{itemize}
    \item 与线性扫描结果对比
    \item 三种索引结果相互对比
    \item 12个单元测试用例全部通过
    \item 在多种数据集上进行验证
\end{itemize}

\subsubsection{理论分析}

从理论角度分析了三种索引的:
\begin{itemize}
    \item 数据划分方式:球形划分 vs 超平面划分 vs 线性划分
    \item 支撑点信息利用:距离值 vs 距离差 vs 距离向量
    \item 剪枝能力:排除规则和包含规则
    \item 时空复杂度:$O(nk\log n)$构建,$O(k\log n + m)$查询
\end{itemize}

\subsubsection{实验分析}

在多种数据集上进行了全面的性能对比:
\begin{itemize}
    \item 低维向量数据集(2维,1000个点)
    \item 高维向量数据集(10维,500个点)
    \item 蛋白质序列数据集(960条序列)
\end{itemize}

主要发现:
\begin{enumerate}
    \item MVP树和LP树性能接近,在大多数场景下表现优异
    \item CGH树在小查询半径时效果较好,但大半径时剪枝效果急剧下降
    \item 高维数据受维度灾难影响,所有索引的剪枝效果都下降
    \item FFT是较好的pivot选择策略,在效果和效率之间取得平衡
\end{enumerate}

\subsection{主要结论}

\begin{enumerate}
    \item \textbf{多Pivot索引的价值}:使用3个pivot可以获得更多的距离信息,实现更有效的剪枝
    
    \item \textbf{划分策略的影响}:不同的划分策略对索引性能有显著影响
    \begin{itemize}
        \item 中位数划分(MVP/LP)保证平衡性
        \item 符号划分(CGH)可能导致不平衡
    \end{itemize}
    
    \item \textbf{包含规则的重要性}:具有包含规则的索引(MVP/LP)在范围查询和kNN查询中表现更好
    
    \item \textbf{维度灾难}:高维数据对所有索引都是挑战,需要更多的pivot或其他技术
    
    \item \textbf{查询半径敏感性}:CGH树对查询半径非常敏感,大半径时几乎无法剪枝
\end{enumerate}

\subsection{不足与改进方向}

\subsubsection{当前实现的不足}

\begin{enumerate}
    \item 只实现了3-pivot版本,没有支持任意数量的pivot
    \item CGH树只实现了4路划分,没有实现8路划分变体
    \item 没有实现动态插入和删除操作
    \item 没有进行磁盘版本的实现和I/O优化
\end{enumerate}

\subsubsection{可能的改进方向}

\begin{enumerate}
    \item \textbf{增加pivot数量}:研究4-pivot或更多pivot的索引效果
    \item \textbf{自适应划分}:根据数据分布自动选择划分策略
    \item \textbf{混合索引}:结合不同划分策略的优点
    \item \textbf{并行化}:利用多核CPU加速构建和查询
    \item \textbf{近似查询}:牺牲部分精度换取更高的查询效率
\end{enumerate}

\subsection{展望}

多Pivot树状索引是度量空间索引的重要研究方向。未来的工作可以从以下几个方面深入:

\begin{enumerate}
    \item \textbf{理论研究}:分析最优pivot数量与数据本征维度的关系
    \item \textbf{实践应用}:将索引应用于实际的相似性搜索场景
    \item \textbf{系统优化}:开发高效的磁盘版本索引系统
    \item \textbf{机器学习结合}:利用机器学习方法优化pivot选择和查询路由
\end{enumerate}

\subsection{致谢}

感谢毛睿教授的悉心指导,感谢大数据泛构课程提供的理论基础和实践平台。

\subsection{参考文献说明}

本实验报告参考了以下主要文献:
\begin{enumerate}
    \item Bozkaya T, Ozsoyoglu M. Indexing large metric spaces for similarity search queries. ACM TODS, 1999.
    \item Mao R, et al. On data partitioning in tree structure metric-space indexes. DASFAA, 2014.
    \item Chávez E, et al. Searching in metric spaces. ACM Computing Surveys, 2001.
    \item 毛睿. 大数据泛构(课程教材).
\end{enumerate}
