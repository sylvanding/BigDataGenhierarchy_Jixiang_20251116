% Chapter 4: Correctness Verification
\section{正确性验证}

\subsection{测试数据集}

为了验证三种多Pivot树索引的正确性,我们使用以下测试数据集:

\subsubsection{小规模2D向量数据集}

用于手工验证的小规模数据集,包含15个2维点:

\begin{table}[htbp]
    \centering
    \caption{小规模测试数据集}
    \label{tab:test-dataset}
    \begin{tabular}{ccc|ccc}
        \toprule
        \textbf{ID} & \textbf{x} & \textbf{y} & \textbf{ID} & \textbf{x} & \textbf{y} \\
        \midrule
        0 & 1.0 & 1.0 & 8 & 9.0 & 4.0 \\
        1 & 2.0 & 2.0 & 9 & 10.0 & 7.0 \\
        2 & 3.0 & 1.0 & 10 & 3.0 & 5.0 \\
        3 & 4.0 & 4.0 & 11 & 6.0 & 1.0 \\
        4 & 5.0 & 2.0 & 12 & 2.0 & 7.0 \\
        5 & 6.0 & 5.0 & 13 & 8.0 & 2.0 \\
        6 & 7.0 & 3.0 & 14 & 4.0 & 8.0 \\
        7 & 8.0 & 6.0 & & & \\
        \bottomrule
    \end{tabular}
\end{table}

\subsubsection{较大规模测试数据集}

用于综合测试的数据集:
\begin{itemize}
    \item 低维向量数据(2维):200个聚类分布点
    \item 高维向量数据(10维):500个均匀分布点
    \item 蛋白质序列数据:960条酵母蛋白质序列(长度6)
\end{itemize}

\subsection{各索引正确性验证}

\subsubsection{MVP树构建与查询验证}

使用小规模数据集验证MVP树的构建和查询正确性。

\textbf{构建结果}:
\begin{verbatim}
=== MVP Tree Build Result ===
Tree Height: 2
Total Nodes: 7
  - Internal Nodes: 3
  - Leaf Nodes: 4
Build Distance Computations: 84

Tree Structure:
└── MVP Internal [pivots=ID5,ID0,ID9, sizes=[2,0,5,0,1,4,0,0]]
    ├── Leaf [size=2]
    ├── MVP Internal [pivots=ID13,ID14,ID7, sizes=[2,0,0,0,0,0,0,0]]
    │   └── Leaf [size=2]
    ├── Leaf [size=1]
    └── MVP Internal [pivots=ID11,ID12,ID1, sizes=[1,0,0,0,0,0,0,0]]
        └── Leaf [size=1]
\end{verbatim}

\textbf{范围查询验证}(查询点(5.0, 5.0),半径3.0):

\begin{table}[htbp]
    \centering
    \caption{MVP树范围查询结果}
    \label{tab:mvp-range-result}
    \begin{tabular}{cccc}
        \toprule
        \textbf{ID} & \textbf{Coordinate} & \textbf{Distance} & \textbf{Result} \\
        \midrule
        5 & (6.0, 5.0) & 1.0000 & \cmark \\
        3 & (4.0, 4.0) & 1.4142 & \cmark \\
        10 & (3.0, 5.0) & 2.0000 & \cmark \\
        6 & (7.0, 3.0) & 2.8284 & \cmark \\
        4 & (5.0, 2.0) & 3.0000 & \cmark \\
        \bottomrule
    \end{tabular}
\end{table}

\subsubsection{CGH树构建与查询验证}

\textbf{构建结果}:
\begin{verbatim}
=== CGH Tree Build Result ===
Tree Height: 2
Total Nodes: 7
  - Internal Nodes: 2
  - Leaf Nodes: 5
Build Distance Computations: 91

Tree Structure:
└── CGH Internal [pivots=ID5,ID0,ID9, sizes=[8,2,2,0]]
    ├── CGH Internal [pivots=ID3,ID13,ID14, sizes=[2,2,1,0]]
    │   ├── Leaf [size=2]
    │   ├── Leaf [size=2]
    │   └── Leaf [size=1]
    ├── Leaf [size=2]
    └── Leaf [size=2]
\end{verbatim}

\textbf{范围查询验证}:结果与MVP树一致,查询距离计算次数为13次。

\subsubsection{线性划分树构建与查询验证}

\textbf{构建结果}:
\begin{verbatim}
=== Linear Partition Tree Build Result ===
Tree Height: 2
Total Nodes: 7
  - Internal Nodes: 3
  - Leaf Nodes: 4
Build Distance Computations: 84

Tree Structure:
└── LP Internal [pivots=ID5,ID0,ID9, sizes=[2,0,5,0,1,4,0,0]]
    ├── Leaf [size=2]
    ├── LP Internal [pivots=ID13,ID14,ID7, sizes=[2,0,0,0,0,0,0,0]]
    │   └── Leaf [size=2]
    ├── Leaf [size=1]
    └── LP Internal [pivots=ID11,ID12,ID1, sizes=[1,0,0,0,0,0,0,0]]
        └── Leaf [size=1]
\end{verbatim}

\textbf{范围查询验证}:结果与前两种索引一致,查询距离计算次数为15次。

\subsection{结果一致性验证}

\subsubsection{与线性扫描对比}

使用线性扫描作为基准,验证三种索引的查询结果一致性:

\begin{table}[htbp]
    \centering
    \caption{范围查询结果一致性验证}
    \label{tab:consistency-range}
    \begin{tabular}{lccc}
        \toprule
        \textbf{Method} & \textbf{Result Count} & \textbf{Distance Computations} & \textbf{Consistent} \\
        \midrule
        Linear Scan & 5 & 15 & - \\
        MVP Tree & 5 & 15 & \cmark \\
        CGH Tree & 5 & 13 & \cmark \\
        LP Tree & 5 & 15 & \cmark \\
        \bottomrule
    \end{tabular}
\end{table}

\subsubsection{kNN查询一致性验证}

对kNN查询(k=3)进行验证:

\begin{table}[htbp]
    \centering
    \caption{kNN查询结果一致性验证}
    \label{tab:consistency-knn}
    \begin{tabular}{lcccc}
        \toprule
        \textbf{Method} & \textbf{1st} & \textbf{2nd} & \textbf{3rd} & \textbf{Consistent} \\
        \midrule
        Linear Scan & ID5 (1.00) & ID3 (1.41) & ID10 (2.00) & - \\
        MVP Tree & ID5 (1.00) & ID3 (1.41) & ID10 (2.00) & \cmark \\
        CGH Tree & ID5 (1.00) & ID3 (1.41) & ID10 (2.00) & \cmark \\
        LP Tree & ID5 (1.00) & ID3 (1.41) & ID10 (2.00) & \cmark \\
        \bottomrule
    \end{tabular}
\end{table}

\subsubsection{大规模数据集验证}

在200个数据点的测试集上进行10轮随机查询验证:

\begin{table}[htbp]
    \centering
    \caption{大规模数据集验证结果}
    \label{tab:large-scale-verify}
    \begin{tabular}{lccc}
        \toprule
        \textbf{Index} & \textbf{Height} & \textbf{Nodes} & \textbf{Correctness} \\
        \midrule
        MVP Tree & 2 & 56 & 10/10 (100\%) \\
        CGH Tree & 5 & 55 & 10/10 (100\%) \\
        LP Tree & 2 & 56 & 10/10 (100\%) \\
        \bottomrule
    \end{tabular}
\end{table}

\subsection{单元测试结果}

运行\texttt{mvn test -Dtest=MultiPivotTreeTest},所有12个测试用例全部通过:

\begin{table}[htbp]
    \centering
    \caption{单元测试结果汇总}
    \label{tab:unit-test}
    \begin{tabular}{clc}
        \toprule
        \textbf{No.} & \textbf{Test Case} & \textbf{Status} \\
        \midrule
        1 & MVP Tree Build & \cmark \\
        2 & MVP Tree Range Query & \cmark \\
        3 & MVP Tree kNN Query & \cmark \\
        4 & CGH Tree Build & \cmark \\
        5 & CGH Tree Range Query & \cmark \\
        6 & CGH Tree kNN Query & \cmark \\
        7 & LP Tree Build & \cmark \\
        8 & LP Tree Range Query & \cmark \\
        9 & LP Tree kNN Query & \cmark \\
        10 & Three Index Consistency & \cmark \\
        11 & Large Dataset Test & \cmark \\
        12 & Empty Dataset Handling & \cmark \\
        \bottomrule
    \end{tabular}
\end{table}

测试运行时间:0.115秒,所有测试通过(Tests run: 12, Failures: 0, Errors: 0)。

\subsection{验证结论}

通过以上验证,我们确认:
\begin{enumerate}
    \item 三种多Pivot树索引均能正确构建,树结构符合预期
    \item 范围查询结果与线性扫描完全一致
    \item kNN查询结果与线性扫描完全一致
    \item 三种索引之间的结果相互一致
    \item 边界情况(空数据集)处理正确
\end{enumerate}
