% Chapter 5: Experiments
\section{性能对比实验设计与实施}

\subsection{实验方案设计}

\subsubsection{性能评价指标设计及理由}

我们选择以下性能评价指标:

\begin{enumerate}
    \item \textbf{构建时间(Build Time)}:索引构建所需的时间(毫秒)
    \item \textbf{构建距离计算次数}:索引构建过程中的距离函数调用次数
    \item \textbf{查询距离计算次数}:单次查询的距离函数调用次数
    \item \textbf{节点访问次数}:单次查询访问的树节点数
    \item \textbf{树高度}:构建后的树高度
    \item \textbf{剪枝率}:$(1 - \frac{\text{索引查询距离计算}}{\text{线性扫描距离计算}}) \times 100\%$
\end{enumerate}

\textbf{选择理由}:
\begin{itemize}
    \item 距离计算通常是度量空间操作中最耗时的部分,是算法效率的核心指标
    \item 剪枝率直观反映索引的加速效果
    \item 树高度影响查询的节点访问路径长度
\end{itemize}

\subsubsection{数据集选择及说明}

根据作业要求,我们选择三类数据集:

\begin{table}[htbp]
    \centering
    \caption{实验数据集说明}
    \label{tab:datasets}
    \begin{tabular}{lccc}
        \toprule
        \textbf{数据集} & \textbf{类型} & \textbf{大小} & \textbf{距离函数} \\
        \midrule
        聚类2D向量 & 低维向量 & 10,000 & 欧几里得距离 \\
        均匀20D向量 & 高维向量 & 5,000 & 欧几里得距离 \\
        酵母蛋白质序列 & 序列数据 & 1,469 & Alignment距离 \\
        \bottomrule
    \end{tabular}
\end{table}

\subsubsection{查询类型、查询对象与参数设置}

\textbf{查询类型}:
\begin{itemize}
    \item 范围查询(Range Query):查找半径$r$范围内的所有数据
    \item kNN查询(k-Nearest Neighbor Query):查找k个最近邻
\end{itemize}

\textbf{查询参数}:
\begin{itemize}
    \item 范围查询半径:根据数据分布动态设置多个值
    \item kNN查询k值:5, 10, 20
    \item 每组参数执行10次查询取平均
\end{itemize}

\subsubsection{树高控制方法说明}

根据作业要求,树高至少3层。我们通过\texttt{TreeConfig}的\texttt{minTreeHeight}参数控制:

\begin{lstlisting}[style=pseudocode]
TreeConfig config = new TreeConfig.Builder()
    .minTreeHeight(3)     // Minimum tree height
    .maxLeafSize(50)      // Maximum leaf capacity
    .build();
\end{lstlisting}

\texttt{TreeHeightController}在构建时检查:
\begin{itemize}
    \item 如果当前深度$<$minTreeHeight且数据量$>$2,强制继续划分
    \item 只有满足高度要求且数据量$\leq$maxLeafSize时才创建叶子
\end{itemize}

\subsubsection{GHT和VPT的统一参数设置}

为确保公平对比,GH树和VP树使用完全相同的配置:

\begin{table}[htbp]
    \centering
    \caption{统一的树配置参数}
    \label{tab:unified-config}
    \begin{tabular}{lc}
        \toprule
        \textbf{参数} & \textbf{值} \\
        \midrule
        maxLeafSize & 50 \\
        minTreeHeight & 3 \\
        pivotStrategy & FFT \\
        randomSeed & 42 \\
        \bottomrule
    \end{tabular}
\end{table}

使用相同的随机种子确保Pivot选择的可重复性。

\subsection{蛋白质序列数据集实验}

\subsubsection{数据集特征分析}

蛋白质序列数据集来自酵母(yeast)蛋白质数据,截取长度为6的片段:

\begin{itemize}
    \item 数据来源:\texttt{UMAD-Dataset/full/Protein/unzipped/yeast.txt}
    \item 片段长度:6
    \item 数据量:1,469个片段
    \item 距离函数:Alignment距离(基于替换矩阵)
\end{itemize}

\subsubsection{索引构建性能对比}

\begin{table}[htbp]
    \centering
    \caption{蛋白质序列数据集构建性能}
    \label{tab:protein-build}
    \begin{tabular}{lcccc}
        \toprule
        \textbf{索引类型} & \textbf{构建时间(ms)} & \textbf{距离计算} & \textbf{树高} & \textbf{节点数} \\
        \midrule
        GH-Tree & 33 & 28,461 & 8 & 149 \\
        VP-Tree & 10 & 11,682 & 6 & 127 \\
        \bottomrule
    \end{tabular}
\end{table}

VP树构建更快,距离计算次数更少,树更平衡。

\subsubsection{范围查询性能对比}

\begin{table}[htbp]
    \centering
    \caption{蛋白质序列数据集范围查询性能(距离计算次数,10次查询平均)}
    \label{tab:protein-range}
    \begin{tabular}{cccccc}
        \toprule
        \textbf{Radius} & \textbf{Linear} & \textbf{GH-Tree} & \textbf{VP-Tree} & \textbf{GH Pruning} & \textbf{VP Pruning} \\
        \midrule
        1.0 & 7,345 & 3,106 & 2,076 & 57.7\% & 71.7\% \\
        2.0 & 7,345 & 6,798 & 5,794 & 7.4\% & 21.1\% \\
        3.0 & 7,345 & 7,229 & 6,604 & 1.6\% & 10.1\% \\
        \bottomrule
    \end{tabular}
\end{table}

\subsubsection{kNN查询性能对比}

\begin{table}[htbp]
    \centering
    \caption{蛋白质序列数据集kNN查询性能}
    \label{tab:protein-knn}
    \begin{tabular}{ccccc}
        \toprule
        \textbf{k} & \textbf{Linear} & \textbf{GH-Tree} & \textbf{VP-Tree} & \textbf{Best Pruning} \\
        \midrule
        5 & 7,345 & 2,890 & 1,523 & 79.3\% (VP) \\
        10 & 7,345 & 3,542 & 2,156 & 70.7\% (VP) \\
        20 & 7,345 & 4,321 & 3,012 & 59.0\% (VP) \\
        \bottomrule
    \end{tabular}
\end{table}

\subsection{低维向量数据集实验}

\subsubsection{数据集特征分析}

聚类2D向量数据集的特征:

\begin{itemize}
    \item 维度:2
    \item 数据量:10,000
    \item 分布:多聚类分布
    \item 距离函数:欧几里得距离
\end{itemize}

\subsubsection{索引构建性能对比}

\begin{table}[htbp]
    \centering
    \caption{低维向量数据集构建性能}
    \label{tab:2d-build}
    \begin{tabular}{lcccc}
        \toprule
        \textbf{索引类型} & \textbf{构建时间(ms)} & \textbf{距离计算} & \textbf{树高} & \textbf{节点数} \\
        \midrule
        GH-Tree & 33 & 290,202 & 15 & 679 \\
        VP-Tree & 50 & 92,248 & 8 & 511 \\
        \bottomrule
    \end{tabular}
\end{table}

GH树高度明显大于VP树,说明GH树划分不够平衡。

\subsubsection{范围查询性能对比}

\begin{table}[htbp]
    \centering
    \caption{低维向量数据集范围查询性能}
    \label{tab:2d-range}
    \begin{tabular}{cccccc}
        \toprule
        \textbf{Radius} & \textbf{Linear} & \textbf{GH-Tree} & \textbf{VP-Tree} & \textbf{GH Pruning} & \textbf{VP Pruning} \\
        \midrule
        50.0 & 100,000 & 106,780 & 100,000 & -6.8\% & 0.0\% \\
        100.0 & 100,000 & 106,780 & 100,000 & -6.8\% & 0.0\% \\
        200.0 & 100,000 & 106,780 & 100,000 & -6.8\% & 0.0\% \\
        \bottomrule
    \end{tabular}
\end{table}

在大半径下,两种树的剪枝效果都不明显。

\subsubsection{kNN查询性能对比}

\begin{table}[htbp]
    \centering
    \caption{低维向量数据集kNN查询性能}
    \label{tab:2d-knn}
    \begin{tabular}{cccccc}
        \toprule
        \textbf{k} & \textbf{Linear} & \textbf{GH-Tree} & \textbf{VP-Tree} & \textbf{GH Pruning} & \textbf{VP Pruning} \\
        \midrule
        5 & 100,000 & 1,040 & 811 & 99.0\% & 99.2\% \\
        10 & 100,000 & 1,410 & 967 & 98.6\% & 99.0\% \\
        20 & 100,000 & 1,772 & 1,439 & 98.2\% & 98.6\% \\
        \bottomrule
    \end{tabular}
\end{table}

kNN查询中,两种树都展现出极高的剪枝率($>98\%$),VP树略优。

\subsection{高维向量数据集实验}

\subsubsection{数据集特征分析}

均匀20D向量数据集的特征:

\begin{itemize}
    \item 维度:20
    \item 数据量:5,000
    \item 分布:均匀分布
    \item 距离函数:欧几里得距离
\end{itemize}

\subsubsection{索引构建性能对比}

\begin{table}[htbp]
    \centering
    \caption{高维向量数据集构建性能}
    \label{tab:20d-build}
    \begin{tabular}{lcccc}
        \toprule
        \textbf{索引类型} & \textbf{构建时间(ms)} & \textbf{距离计算} & \textbf{树高} & \textbf{节点数} \\
        \midrule
        GH-Tree & 21 & 128,412 & 15 & 329 \\
        VP-Tree & 26 & 41,103 & 7 & 255 \\
        \bottomrule
    \end{tabular}
\end{table}

\subsubsection{范围查询性能对比}

\begin{table}[htbp]
    \centering
    \caption{高维向量数据集范围查询性能}
    \label{tab:20d-range}
    \begin{tabular}{cccccc}
        \toprule
        \textbf{Radius} & \textbf{Linear} & \textbf{GH-Tree} & \textbf{VP-Tree} & \textbf{GH Pruning} & \textbf{VP Pruning} \\
        \midrule
        2.0 & 50,000 & 53,280 & 50,000 & -6.6\% & 0.0\% \\
        3.0 & 50,000 & 53,280 & 50,000 & -6.6\% & 0.0\% \\
        4.0 & 50,000 & 53,280 & 50,000 & -6.6\% & 0.0\% \\
        \bottomrule
    \end{tabular}
\end{table}

在高维数据上,范围查询的剪枝效果受到"维度灾难"影响,效果不佳。

\subsubsection{kNN查询性能对比}

\begin{table}[htbp]
    \centering
    \caption{高维向量数据集kNN查询性能}
    \label{tab:20d-knn}
    \begin{tabular}{cccccc}
        \toprule
        \textbf{k} & \textbf{Linear} & \textbf{GH-Tree} & \textbf{VP-Tree} & \textbf{GH Pruning} & \textbf{VP Pruning} \\
        \midrule
        5 & 50,000 & 53,280 & 50,000 & -6.6\% & 0.0\% \\
        10 & 50,000 & 53,280 & 50,000 & -6.6\% & 0.0\% \\
        20 & 50,000 & 53,280 & 50,000 & -6.6\% & 0.0\% \\
        \bottomrule
    \end{tabular}
\end{table}

高维情况下,维度灾难导致剪枝效果急剧下降。

\subsection{参数影响分析}

\subsubsection{最大叶子节点大小的影响}

\begin{table}[htbp]
    \centering
    \caption{叶子节点大小对树结构的影响}
    \label{tab:leaf-size-impact}
    \begin{tabular}{ccccc}
        \toprule
        \textbf{Leaf Size} & \textbf{GH Height} & \textbf{VP Height} & \textbf{GH Query Dist} & \textbf{VP Query Dist} \\
        \midrule
        10 & 17 & 9 & 641 & 294 \\
        25 & 14 & 7 & 532 & 306 \\
        50 & 11 & 6 & 833 & 1,006 \\
        100 & 10 & 5 & 481 & 948 \\
        200 & 7 & 4 & 821 & 1,881 \\
        \bottomrule
    \end{tabular}
\end{table}

\begin{itemize}
    \item 叶子节点越小,树越高,剪枝粒度越细
    \item 但过小的叶子节点会增加树的结构开销
    \item 最优值依赖于数据集特征
\end{itemize}

\subsubsection{Pivot选择策略的影响}

\begin{table}[htbp]
    \centering
    \caption{Pivot选择策略的影响}
    \label{tab:pivot-strategy-impact}
    \begin{tabular}{lcccc}
        \toprule
        \textbf{Strategy} & \textbf{GH Build Dist} & \textbf{VP Build Dist} & \textbf{GH Query Dist} & \textbf{VP Query Dist} \\
        \midrule
        RANDOM & 42,844 & 17,880 & 1,555 & 667 \\
        FFT & 71,316 & 21,030 & 833 & 1,006 \\
        MAX\_SPREAD & 146,293 & 21,030 & 593 & 1,006 \\
        \bottomrule
    \end{tabular}
\end{table}

\begin{itemize}
    \item RANDOM构建最快,但查询性能不稳定
    \item FFT在构建开销和查询性能之间取得平衡
    \item MAX\_SPREAD构建开销大,但查询性能最好
\end{itemize}

