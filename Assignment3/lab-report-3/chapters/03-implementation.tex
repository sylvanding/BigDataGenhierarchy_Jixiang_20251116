% Chapter 3: Implementation
\section{GHT和VPT实现}

\subsection{系统架构扩展}

\subsubsection{在原有架构中集成树状索引模块}

在Assignment 1和2的基础上,我们扩展系统架构以支持树状索引。图\ref{fig:architecture}展示了扩展后的系统模块关系。

\begin{figure}[htbp]
    \centering
    \begin{tikzpicture}[
        node distance=1.5cm,
        box/.style={rectangle, draw, rounded corners, minimum width=2.5cm, minimum height=0.8cm, text centered, font=\small},
        arrow/.style={->, thick}
    ]
        % Core layer
        \node[box, fill=blue!20] (metric) {MetricSpaceData};
        \node[box, fill=blue!20, right=0.5cm of metric] (func) {MetricFunction};
        
        % Data types
        \node[box, fill=green!20, below left=1cm and -0.5cm of metric] (vector) {VectorData};
        \node[box, fill=green!20, right=0.3cm of vector] (protein) {ProteinData};
        
        % Index layer
        \node[box, fill=orange!20, below=2.5cm of metric] (index) {Index Interface};
        
        % Tree index
        \node[box, fill=yellow!20, below left=1cm and -0.5cm of index] (treeindex) {TreeIndex};
        \node[box, fill=red!20, below left=0.8cm and -0.3cm of treeindex] (ghtree) {GHTree};
        \node[box, fill=red!20, right=0.3cm of ghtree] (vptree) {VPTree};
        
        % Pivot table
        \node[box, fill=yellow!20, below right=1cm and -0.5cm of index] (pivot) {PivotTable};
        
        % Arrows
        \draw[arrow] (vector) -- (metric);
        \draw[arrow] (protein) -- (metric);
        \draw[arrow] (treeindex) -- (index);
        \draw[arrow] (pivot) -- (index);
        \draw[arrow] (ghtree) -- (treeindex);
        \draw[arrow] (vptree) -- (treeindex);
        
        % Labels
        \node[above=0.3cm of metric, font=\footnotesize\bfseries] {Core Layer (Assignment 1)};
        \node[below=0.2cm of index, font=\footnotesize\bfseries, xshift=-2cm] {Tree Index (Assignment 3)};
    \end{tikzpicture}
    \caption{系统架构图}
    \label{fig:architecture}
\end{figure}

\subsubsection{更新后的系统模块划分}

表\ref{tab:modules}展示了Assignment 3新增的主要模块。

\begin{table}[htbp]
    \centering
    \caption{Assignment 3新增模块}
    \label{tab:modules}
    \begin{tabular}{ll}
        \toprule
        \textbf{模块/类} & \textbf{功能说明} \\
        \midrule
        \texttt{index.Index} & 所有索引结构的统一接口 \\
        \texttt{index.tree.TreeIndex} & 树索引的抽象基类 \\
        \texttt{index.tree.TreeNode} & 树节点接口 \\
        \texttt{index.tree.InternalNode} & 内部节点抽象类 \\
        \texttt{index.tree.LeafNode} & 叶子节点类 \\
        \texttt{index.tree.common.TreeConfig} & 树配置类 \\
        \texttt{index.tree.common.TreeHeightController} & 树高控制器 \\
        \texttt{index.tree.ghtree.GHTree} & GH树实现 \\
        \texttt{index.tree.ghtree.GHInternalNode} & GH树内部节点 \\
        \texttt{index.tree.vptree.VPTree} & VP树实现 \\
        \texttt{index.tree.vptree.VPInternalNode} & VP树内部节点 \\
        \bottomrule
    \end{tabular}
\end{table}

\subsection{GHT实现}

\subsubsection{GHT节点数据结构实现}

GH树内部节点的核心实现如下:

\begin{lstlisting}[caption=GHInternalNode核心代码]
public class GHInternalNode extends InternalNode {
    public GHInternalNode(MetricSpaceData pivot1, 
                          MetricSpaceData pivot2,
                          TreeNode leftChild, 
                          TreeNode rightChild, 
                          int depth) {
        this.pivots = List.of(pivot1, pivot2);
        this.children = List.of(leftChild, rightChild);
        this.depth = depth;
    }
    
    public MetricSpaceData getPivot1() { return pivots.get(0); }
    public MetricSpaceData getPivot2() { return pivots.get(1); }
    public TreeNode getLeftChild() { return children.get(0); }
    public TreeNode getRightChild() { return children.get(1); }
}
\end{lstlisting}

\subsubsection{GHT批建算法实现}

GH树批建的核心逻辑实现如下:

\begin{lstlisting}[caption=GH树批建算法核心实现]
protected TreeNode buildTreeRecursive(
        List<? extends MetricSpaceData> data, int depth) {
    // Step 1: Check termination condition
    if (heightController.canCreateLeaf(depth, data.size())) {
        return new LeafNode(data, depth);
    }
    
    // Step 2: Select two pivots
    MetricSpaceData[] pivots = selectTwoPivots(data);
    MetricSpaceData pivot1 = pivots[0];
    MetricSpaceData pivot2 = pivots[1];
    
    // Step 3: Partition data
    List<MetricSpaceData> leftData = new ArrayList<>();
    List<MetricSpaceData> rightData = new ArrayList<>();
    
    for (MetricSpaceData obj : data) {
        double d1 = metric.getDistance(obj, pivot1);
        double d2 = metric.getDistance(obj, pivot2);
        buildDistanceComputations += 2;
        
        if (d1 < d2) {
            leftData.add(obj);
        } else {
            rightData.add(obj);
        }
    }
    
    // Step 4: Recursive build
    TreeNode leftChild = buildTreeRecursive(leftData, depth + 1);
    TreeNode rightChild = buildTreeRecursive(rightData, depth + 1);
    
    return new GHInternalNode(pivot1, pivot2, 
                              leftChild, rightChild, depth);
}
\end{lstlisting}

\subsubsection{GHT范围查询实现}

GH树范围查询实现了超平面剪枝规则:

\begin{lstlisting}[caption=GH树范围查询实现]
public List<MetricSpaceData> rangeQuery(
        MetricSpaceData queryObject, double radius) {
    List<MetricSpaceData> result = new ArrayList<>();
    resetStatistics();
    rangeQueryRecursive(root, queryObject, radius, result);
    return result;
}

private void rangeQueryRecursive(TreeNode node, 
        MetricSpaceData q, double r, List<MetricSpaceData> result) {
    nodeAccesses++;
    
    if (node.isLeaf()) {
        LeafNode leaf = (LeafNode) node;
        for (MetricSpaceData obj : leaf.getData()) {
            double dist = metric.getDistance(q, obj);
            queryDistanceComputations++;
            if (dist <= r) {
                result.add(obj);
            }
        }
        return;
    }
    
    GHInternalNode internal = (GHInternalNode) node;
    double d1 = metric.getDistance(q, internal.getPivot1());
    double d2 = metric.getDistance(q, internal.getPivot2());
    queryDistanceComputations += 2;
    
    // Pruning rule 1: if d1 - d2 > 2r, prune left
    if (!(d1 - d2 > 2 * r)) {
        rangeQueryRecursive(internal.getLeftChild(), q, r, result);
    }
    
    // Pruning rule 2: if d2 - d1 > 2r, prune right
    if (!(d2 - d1 > 2 * r)) {
        rangeQueryRecursive(internal.getRightChild(), q, r, result);
    }
}
\end{lstlisting}

\subsubsection{GHT kNN查询实现}

kNN查询使用优先队列维护当前k个最近邻:

\begin{lstlisting}[caption=GH树kNN查询实现]
public List<MetricSpaceData> knnQuery(
        MetricSpaceData queryObject, int k) {
    PriorityQueue<DataWithDistance> knnQueue = 
        new PriorityQueue<>((a, b) -> 
            Double.compare(b.distance, a.distance)); // max-heap
    
    resetStatistics();
    knnQueryRecursive(root, queryObject, k, knnQueue);
    
    // Convert to result list
    List<MetricSpaceData> result = new ArrayList<>();
    while (!knnQueue.isEmpty()) {
        result.add(0, knnQueue.poll().data);
    }
    return result;
}
\end{lstlisting}

\subsection{VPT实现}

\subsubsection{VPT节点数据结构实现}

VP树内部节点存储距离范围信息:

\begin{lstlisting}[caption=VPInternalNode核心代码]
public class VPInternalNode extends InternalNode {
    private List<DistanceRange> distanceRanges;
    
    public static class DistanceRange {
        public double lower;  // minimum distance
        public double upper;  // maximum distance
        
        public DistanceRange(double lower, double upper) {
            this.lower = lower;
            this.upper = upper;
        }
    }
    
    public VPInternalNode(MetricSpaceData pivot,
                          List<TreeNode> children,
                          List<DistanceRange> distanceRanges,
                          int depth) {
        this.pivots = Collections.singletonList(pivot);
        this.children = new ArrayList<>(children);
        this.distanceRanges = new ArrayList<>(distanceRanges);
        this.depth = depth;
    }
    
    public MetricSpaceData getPivot() { return pivots.get(0); }
    public DistanceRange getDistanceRange(int i) { 
        return distanceRanges.get(i); 
    }
}
\end{lstlisting}

\subsubsection{VPT批建算法实现}

VP树批建基于中位数划分:

\begin{lstlisting}[caption=VP树批建算法核心实现]
protected TreeNode buildTreeRecursive(
        List<? extends MetricSpaceData> data, int depth) {
    if (heightController.canCreateLeaf(depth, data.size())) {
        return new LeafNode(data, depth);
    }
    
    // Select pivot
    MetricSpaceData pivot = selectPivot(data);
    
    // Compute distances and sort
    List<DataWithDistance> dataWithDist = new ArrayList<>();
    for (MetricSpaceData obj : data) {
        if (obj == pivot) continue;
        double dist = metric.getDistance(obj, pivot);
        buildDistanceComputations++;
        dataWithDist.add(new DataWithDistance(obj, dist));
    }
    dataWithDist.sort(Comparator.comparingDouble(d -> d.distance));
    
    // Partition by median
    int midpoint = dataWithDist.size() / 2;
    List<MetricSpaceData> innerData = new ArrayList<>();
    List<MetricSpaceData> outerData = new ArrayList<>();
    
    for (int i = 0; i < dataWithDist.size(); i++) {
        if (i < midpoint) {
            innerData.add(dataWithDist.get(i).data);
        } else {
            outerData.add(dataWithDist.get(i).data);
        }
    }
    
    // Compute distance ranges
    double innerLower = dataWithDist.get(0).distance;
    double innerUpper = dataWithDist.get(midpoint - 1).distance;
    double outerLower = dataWithDist.get(midpoint).distance;
    double outerUpper = dataWithDist.get(dataWithDist.size()-1).distance;
    
    // Recursive build
    TreeNode innerChild = buildTreeRecursive(innerData, depth + 1);
    TreeNode outerChild = buildTreeRecursive(outerData, depth + 1);
    
    List<DistanceRange> ranges = Arrays.asList(
        new DistanceRange(innerLower, innerUpper),
        new DistanceRange(outerLower, outerUpper)
    );
    
    return new VPInternalNode(pivot, 
        Arrays.asList(innerChild, outerChild), ranges, depth);
}
\end{lstlisting}

\subsubsection{VPT范围查询实现}

VP树范围查询利用距离范围剪枝:

\begin{lstlisting}[caption=VP树范围查询实现]
private void rangeQueryRecursive(TreeNode node, 
        MetricSpaceData q, double r, List<MetricSpaceData> result) {
    nodeAccesses++;
    
    if (node.isLeaf()) {
        LeafNode leaf = (LeafNode) node;
        for (MetricSpaceData obj : leaf.getData()) {
            double dist = metric.getDistance(q, obj);
            queryDistanceComputations++;
            if (dist <= r) {
                result.add(obj);
            }
        }
        return;
    }
    
    VPInternalNode internal = (VPInternalNode) node;
    double dq = metric.getDistance(q, internal.getPivot());
    queryDistanceComputations++;
    
    for (int i = 0; i < internal.getChildren().size(); i++) {
        DistanceRange range = internal.getDistanceRange(i);
        double L = range.lower;
        double U = range.upper;
        
        // Pruning: if query ball doesn't intersect with range
        if (!(dq + r < L || dq - r > U)) {
            rangeQueryRecursive(internal.getChildren().get(i), 
                               q, r, result);
        }
    }
}
\end{lstlisting}

\subsubsection{VPT kNN查询实现}

VP树kNN查询与GH树类似,使用动态半径剪枝。

\subsection{Pivot选择与树高控制}

\subsubsection{统一的Pivot选择策略}

为了公平对比,GH树和VP树使用统一的Pivot选择策略。我们实现了三种策略:

\begin{enumerate}
    \item \textbf{RANDOM}:随机选择,时间复杂度$O(1)$
    \item \textbf{FFT(Farthest-First Traversal)}:选择距离较远的点,时间复杂度$O(n)$
    \item \textbf{MAX\_SPREAD}:采样找最大距离对,时间复杂度$O(k^2)$,$k$为采样数
\end{enumerate}

FFT策略的实现:

\begin{lstlisting}[caption=FFT Pivot选择策略]
private MetricSpaceData[] selectFFTPivots(
        List<? extends MetricSpaceData> data) {
    Random random = new Random(config.getSeed());
    // First pivot: random
    MetricSpaceData pivot1 = data.get(random.nextInt(data.size()));
    
    // Second pivot: farthest from first
    MetricSpaceData pivot2 = null;
    double maxDist = -1;
    for (MetricSpaceData obj : data) {
        double dist = metric.getDistance(pivot1, obj);
        buildDistanceComputations++;
        if (dist > maxDist) {
            maxDist = dist;
            pivot2 = obj;
        }
    }
    return new MetricSpaceData[] {pivot1, pivot2};
}
\end{lstlisting}

\subsubsection{树高控制方法}

根据作业要求,树高至少为3层。我们通过TreeHeightController控制:

\begin{lstlisting}[caption=树高控制器实现]
public class TreeHeightController {
    private TreeConfig config;
    
    public boolean canCreateLeaf(int currentDepth, int dataSize) {
        // Must create leaf if data too small
        if (dataSize <= 2) {
            return true;
        }
        
        // Must continue if not reached minimum height
        if (currentDepth < config.getMinTreeHeight()) {
            return false;
        }
        
        // Can create leaf if data size <= maxLeafSize
        return dataSize <= config.getMaxLeafSize();
    }
}
\end{lstlisting}

树配置使用Builder模式,便于统一设置参数:

\begin{lstlisting}[caption=树配置使用示例]
TreeConfig config = new TreeConfig.Builder()
    .maxLeafSize(50)           // max leaf capacity
    .minTreeHeight(3)          // minimum tree height
    .pivotStrategy(TreeConfig.PivotSelectionStrategy.FFT)
    .randomSeed(42)            // fixed seed for reproducibility
    .verbose(false)
    .build();
\end{lstlisting}

